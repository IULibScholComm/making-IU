% Options for packages loaded elsewhere
\PassOptionsToPackage{unicode}{hyperref}
\PassOptionsToPackage{hyphens}{url}
%
\documentclass[
  letterpaper,
]{scrbook}

\usepackage{amsmath,amssymb}
\usepackage{iftex}
\ifPDFTeX
  \usepackage[T1]{fontenc}
  \usepackage[utf8]{inputenc}
  \usepackage{textcomp} % provide euro and other symbols
\else % if luatex or xetex
  \usepackage{unicode-math}
  \defaultfontfeatures{Scale=MatchLowercase}
  \defaultfontfeatures[\rmfamily]{Ligatures=TeX,Scale=1}
\fi
\usepackage{lmodern}
\ifPDFTeX\else  
    % xetex/luatex font selection
\fi
% Use upquote if available, for straight quotes in verbatim environments
\IfFileExists{upquote.sty}{\usepackage{upquote}}{}
\IfFileExists{microtype.sty}{% use microtype if available
  \usepackage[]{microtype}
  \UseMicrotypeSet[protrusion]{basicmath} % disable protrusion for tt fonts
}{}
\makeatletter
\@ifundefined{KOMAClassName}{% if non-KOMA class
  \IfFileExists{parskip.sty}{%
    \usepackage{parskip}
  }{% else
    \setlength{\parindent}{0pt}
    \setlength{\parskip}{6pt plus 2pt minus 1pt}}
}{% if KOMA class
  \KOMAoptions{parskip=half}}
\makeatother
\usepackage{xcolor}
\setlength{\emergencystretch}{3em} % prevent overfull lines
\setcounter{secnumdepth}{5}
% Make \paragraph and \subparagraph free-standing
\makeatletter
\ifx\paragraph\undefined\else
  \let\oldparagraph\paragraph
  \renewcommand{\paragraph}{
    \@ifstar
      \xxxParagraphStar
      \xxxParagraphNoStar
  }
  \newcommand{\xxxParagraphStar}[1]{\oldparagraph*{#1}\mbox{}}
  \newcommand{\xxxParagraphNoStar}[1]{\oldparagraph{#1}\mbox{}}
\fi
\ifx\subparagraph\undefined\else
  \let\oldsubparagraph\subparagraph
  \renewcommand{\subparagraph}{
    \@ifstar
      \xxxSubParagraphStar
      \xxxSubParagraphNoStar
  }
  \newcommand{\xxxSubParagraphStar}[1]{\oldsubparagraph*{#1}\mbox{}}
  \newcommand{\xxxSubParagraphNoStar}[1]{\oldsubparagraph{#1}\mbox{}}
\fi
\makeatother


\providecommand{\tightlist}{%
  \setlength{\itemsep}{0pt}\setlength{\parskip}{0pt}}\usepackage{longtable,booktabs,array}
\usepackage{calc} % for calculating minipage widths
% Correct order of tables after \paragraph or \subparagraph
\usepackage{etoolbox}
\makeatletter
\patchcmd\longtable{\par}{\if@noskipsec\mbox{}\fi\par}{}{}
\makeatother
% Allow footnotes in longtable head/foot
\IfFileExists{footnotehyper.sty}{\usepackage{footnotehyper}}{\usepackage{footnote}}
\makesavenoteenv{longtable}
\usepackage{graphicx}
\makeatletter
\newsavebox\pandoc@box
\newcommand*\pandocbounded[1]{% scales image to fit in text height/width
  \sbox\pandoc@box{#1}%
  \Gscale@div\@tempa{\textheight}{\dimexpr\ht\pandoc@box+\dp\pandoc@box\relax}%
  \Gscale@div\@tempb{\linewidth}{\wd\pandoc@box}%
  \ifdim\@tempb\p@<\@tempa\p@\let\@tempa\@tempb\fi% select the smaller of both
  \ifdim\@tempa\p@<\p@\scalebox{\@tempa}{\usebox\pandoc@box}%
  \else\usebox{\pandoc@box}%
  \fi%
}
% Set default figure placement to htbp
\def\fps@figure{htbp}
\makeatother
% definitions for citeproc citations
\NewDocumentCommand\citeproctext{}{}
\NewDocumentCommand\citeproc{mm}{%
  \begingroup\def\citeproctext{#2}\cite{#1}\endgroup}
\makeatletter
 % allow citations to break across lines
 \let\@cite@ofmt\@firstofone
 % avoid brackets around text for \cite:
 \def\@biblabel#1{}
 \def\@cite#1#2{{#1\if@tempswa , #2\fi}}
\makeatother
\newlength{\cslhangindent}
\setlength{\cslhangindent}{1.5em}
\newlength{\csllabelwidth}
\setlength{\csllabelwidth}{3em}
\newenvironment{CSLReferences}[2] % #1 hanging-indent, #2 entry-spacing
 {\begin{list}{}{%
  \setlength{\itemindent}{0pt}
  \setlength{\leftmargin}{0pt}
  \setlength{\parsep}{0pt}
  % turn on hanging indent if param 1 is 1
  \ifodd #1
   \setlength{\leftmargin}{\cslhangindent}
   \setlength{\itemindent}{-1\cslhangindent}
  \fi
  % set entry spacing
  \setlength{\itemsep}{#2\baselineskip}}}
 {\end{list}}
\usepackage{calc}
\newcommand{\CSLBlock}[1]{\hfill\break\parbox[t]{\linewidth}{\strut\ignorespaces#1\strut}}
\newcommand{\CSLLeftMargin}[1]{\parbox[t]{\csllabelwidth}{\strut#1\strut}}
\newcommand{\CSLRightInline}[1]{\parbox[t]{\linewidth - \csllabelwidth}{\strut#1\strut}}
\newcommand{\CSLIndent}[1]{\hspace{\cslhangindent}#1}

\makeatletter
\@ifpackageloaded{bookmark}{}{\usepackage{bookmark}}
\makeatother
\makeatletter
\@ifpackageloaded{caption}{}{\usepackage{caption}}
\AtBeginDocument{%
\ifdefined\contentsname
  \renewcommand*\contentsname{Table of contents}
\else
  \newcommand\contentsname{Table of contents}
\fi
\ifdefined\listfigurename
  \renewcommand*\listfigurename{List of Figures}
\else
  \newcommand\listfigurename{List of Figures}
\fi
\ifdefined\listtablename
  \renewcommand*\listtablename{List of Tables}
\else
  \newcommand\listtablename{List of Tables}
\fi
\ifdefined\figurename
  \renewcommand*\figurename{Figure}
\else
  \newcommand\figurename{Figure}
\fi
\ifdefined\tablename
  \renewcommand*\tablename{Table}
\else
  \newcommand\tablename{Table}
\fi
}
\@ifpackageloaded{float}{}{\usepackage{float}}
\floatstyle{ruled}
\@ifundefined{c@chapter}{\newfloat{codelisting}{h}{lop}}{\newfloat{codelisting}{h}{lop}[chapter]}
\floatname{codelisting}{Listing}
\newcommand*\listoflistings{\listof{codelisting}{List of Listings}}
\makeatother
\makeatletter
\makeatother
\makeatletter
\@ifpackageloaded{caption}{}{\usepackage{caption}}
\@ifpackageloaded{subcaption}{}{\usepackage{subcaption}}
\makeatother

\usepackage{bookmark}

\IfFileExists{xurl.sty}{\usepackage{xurl}}{} % add URL line breaks if available
\urlstyle{same} % disable monospaced font for URLs
\hypersetup{
  pdftitle={Making Indiana University},
  pdfauthor={James H. Capshew},
  hidelinks,
  pdfcreator={LaTeX via pandoc}}


\title{Making Indiana University}
\usepackage{etoolbox}
\makeatletter
\providecommand{\subtitle}[1]{% add subtitle to \maketitle
  \apptocmd{\@title}{\par {\large #1 \par}}{}{}
}
\makeatother
\subtitle{History, Landscape, and a Sense of Place}
\author{James H. Capshew}
\date{2025}

\begin{document}
\frontmatter
\maketitle
\begin{abstract}
This book sheds light on the creation of Indiana University's
institutional identity and image over its two centuries of existence by
investigating the role of historians, archivists, and others in
documenting its historical record. As such, it is an exercise in
historiography, or the study of the history of IU history. The first
part presents a rationale for an inclusive view of contributors to IU
history, including not only historians and archivists but also
architects, groundskeepers, and other members of its academic community.
Essays on the contributions of the first professor as well as the
invention of the genre of Indiana University history provides a useful
provenance. The second section supplies a chronological study of the
history of IU's distinctive campus design, from its beginning in 1885 to
2020, to illustrate the essential role that place plays in university
culture. The third and final part are essays that uncover hidden efforts
to sustain the university's historical record in publications, faculty
memorials, and historic preservation. By its interrogation of the
sources and methods that construct the historical record, this book
makes a unique contribution to the study of Indiana University history
and culture.
\end{abstract}

\renewcommand*\contentsname{Table of contents}
{
\setcounter{tocdepth}{2}
\tableofcontents
}

\mainmatter
\bookmarksetup{startatroot}

\chapter{Front Matter}\label{front-matter}

\section{Acknowledgements}\label{acknowledgements}

\emph{A scrawny boy, aged 9, pedaled his 20-inch red Schwinn bicycle up
Woodlawn Avenue from Bryan Park to the Indiana University campus,
looking for adventure. The warm spring air was full of mysterious
promise. Accompanied by a friend who lived across the street on Maxwell
Lane and one of his brothers, the boys knew that winding paths through
the woods, immense gray buildings, and running water would await them.
Once on campus grounds, anticipation gave way to delight in the present
moment---riding endlessly on interconnected walkways, catching goldfish
in the creek left discarded from fraternity parties, searching for a
drinking fountain in the cavernous Union building. The waking reverie
lasted until the lengthening shadows brought them back to the hunger in
their bellies and to the relief of their homes. In the spring and summer
of 1964, that boy was introduced to the wonders of university life by
the physical environment of the campus, where he was able to engage all
of his senses, six blocks away from his home. Still innocent of the ways
of place-making, the names Maxwell and Bryan held no connotations. The
boy's name was Jimmy Capshew, and he has lived in the university's
sheltering shadow since.}

This book was a long time in the making, delayed by the global pandemic.
Along the way I received assistance from many people, most notably from
the excellent staff of the University Archives at Indiana University
Bloomington. I am grateful to my colleagues Dina Kellams, Kristin
Leaman, Carrie Schwier, Mary Mellon, Molly Wittenberg, Brad Cook, and
Amanda Rindler.

In 2015, I was drawn into the organization and then operation of the IU
Office of the Bicentennial under its mastermind, Kelly Kish, whose
administrative acumen is matched only by her prodigious intellect. She
always asked the hard questions, but I could count on her for unflagging
support. The staff of the Bicentennial Office, including Bre Anne Kusz,
Jeremy Hackerd, Sarah Jacobi, Angel Nathan, Sarah Reynolds, Rafal
Swiatkowski, and Brittany Terwilliger, buoyed my efforts at every turn.

I made presentations to several audiences, both local and national,
about the research on which this book is based, including the Monroe
County History Club, the IU Alumni Association's Mini University, the
Friends of the Monroe County Public Library, and the History of
Education Society.

Several persons facilitated access to documents and shared relevant
information, including Anita Bracalente, Jonah Busch, Greg Buse, Carey
Champion, Terry Clapacs, Michael Chitwood, Bridget Edwards, Harry Ford,
Deborah Lemon, Richard McClelland, Sarah Mincey, David Parkhurst, Eileen
Savage, and Indermohan Virk.

The people who read drafts of chapters occupy a high niche in my
personal pantheon: Bre Anne Kusz, Jonah Busch, Duncan Campbell, Carey
Champion, Michael Chitwood, Harry Ford, Donald J. Gray, Jeremy Hackerd,
Sarah Mincey, Michael Nelson, Laura Plummer, Sarah Reynolds, Eric
Sandweiss, Curt Simic, and John Summerlot. It was a pleasure working
with illustrator Joe Lee, whom I first met two decades ago.

When the time came to publish this research, Diane Dallis-Comentale,
Ruth Lilly Dean of University Libraries, suggested I try the new
publishing service provided by the Department of Scholarly
Communication. Adam Mazel, Digital Publishing Librarian, was an
excellent guide and an effective colleague in creating my first ``born
digital'' book.

A special thanks goes to Michael McRobbie, University Chancellor and
President Emeritus, who had faith in planning for the university's
future by a thorough understanding of its past.

Halloween 2024

\textbf{How to Report a Problem (Technological, Accessibility, Textual,
Etc.) with this Book}

Click ``Report an issue'' in the right-hand sidebar of the \href{}{HTML
version}; then complete and submit the bug report.

\bookmarksetup{startatroot}

\chapter{Introduction}\label{introduction}

This is a book created from markdown and executable code.

See Donald E. Knuth\footnote{{``Literate Programming,''} \emph{Comput.
  J.} 27, no. 2 (May 1984): 97--111,
  \url{https://doi.org/10.1093/comjnl/27.2.97}.} for additional
discussion of literate programming.

\bookmarksetup{startatroot}

\chapter{Summary}\label{summary}

In summary, this book has no content whatsoever.

\bookmarksetup{startatroot}

\chapter*{References}\label{references}
\addcontentsline{toc}{chapter}{References}

\markboth{References}{References}

\phantomsection\label{refs}
\begin{CSLReferences}{1}{0}
\bibitem[\citeproctext]{ref-knuth84}
Knuth, Donald E. {``Literate Programming.''} \emph{Comput. J.} 27, no. 2
(May 1984): 97--111. \url{https://doi.org/10.1093/comjnl/27.2.97}.

\end{CSLReferences}


\backmatter


\end{document}
